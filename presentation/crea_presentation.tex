\documentclass[mathserif]{beamer}
\usetheme{Berkeley}
\usecolortheme{albatross}
\usepackage{listings}
\usepackage{pgf}
\usepackage{qtree}
\usepackage{tikz}
\usetikzlibrary{shapes, arrows}
\usepackage{gb4e}


\title{Constructing a Knowledge Base on Aging}
\subtitle{An Automated Approach}
\author{Mark Farrell}
\institute{

Bioinformatics Researcher \and

\inst{}%
Center for Research and Education on Aging \\
Lawrence Berkeley National Laboratory \\
University of California, Berkeley

}

\note{
Good afternoon everybody. My name's Mark Farrell and I've been doing research for CREA
remotely from Canada. Today I'd like to inform you about a method I've been working on for
automating the construction of biomedical knowledge bases; specifically a knowledge base on aging.
The presentation should be about 20 minutes in length and there will be an opportunity to ask
questions at the end.
}

\date{September 4th, 2014}
\subject{Bioinformatics}

\AtBeginSection[]
{
\begin{frame}
\frametitle{Outline}
\tableofcontents[currentsection]
\end{frame}
}

\lstset{
basicstyle=\small\sffamily,
columns=fullflexible,
showstringspaces=false
}

\noautomath

%\setbeameroption{show notes}

\begin{document}

\begin{frame}
\titlepage
\end{frame}

\section{About Me}

\begin{frame}

\frametitle{Who I am}
\framesubtitle{Mark Farrell}

\centering

\begin{tabular}{c c}
\includegraphics[width=0.30\linewidth]{images/starfall.png}&
\includegraphics[width=0.30\linewidth]{images/rink.jpg} \\
\includegraphics[angle=270, origin=c, width=0.30\linewidth]{images/alfie.jpg}&
\includegraphics[angle=270, origin=c, width=0.30\linewidth]{images/camping.jpg}\\
\end{tabular}

\end{frame}

\note{
I'm going to start off by introducing myself.
This is my first time being here in person, so I doubt many of you know who I am.
I've lived most of my life in Canada's East Coast. I was back home again this summer,
working as a co-op student for Defence R\&D Canada.
}

\begin{frame}
\frametitle{Where I'm from}
\framesubtitle{Atlantic Canada}
\includegraphics[width=1.0\linewidth]{images/map.png}
\end{frame}

\note{
So, I've had to travel quite a bit over the past few days in order to arrive here at
Berkeley.
}

\begin{frame}

\frametitle{What I'm Studying}

\begin{columns}[l]
\begin{column}{.7\textwidth}
\begin{description}
\item[Year] 2
\item[Program] Bachelor of Computer Science
\item[Faculty] Mathematics
\item[Institution] University of Waterloo
\item[Location] Waterloo, Ontario, Canada
\end{description}
\end{column}
\begin{column}{.3\textwidth}
\includegraphics[width=1.0\linewidth]{images/campus.jpg}
\end{column}
\end{columns}

\end{frame}

\note{
And in case you're wondering, I'm going into my second year of undergraduate studies
at the University of Waterloo, majoring in Computer Science.
}

\begin{frame}

\frametitle{How I Became a Researcher at CREA}
\framesubtitle{Prof. Garan Emailed the University of Waterloo Computer Science Club}

\centering

\begin{tabular}{c c}
\includegraphics[width=0.40\linewidth]{images/malto_webcam.png} &
\includegraphics[width=0.40\linewidth]{images/bitshift_webcam.png}\\
\includegraphics[width=0.40\linewidth]{images/csc.png} &
\includegraphics[width=0.40\linewidth]{images/steve_garan.jpg}\\
\end{tabular}

\end{frame}

\note{
That's all very well and good: I haven't explained how I got involved
with research at CREA. Steve Garan (you might know him) emailed the
University of Waterloo Computer Science Club back in January, looking for
current members interested in an AI internship; he used to
be President back in the 80s. It looked interesting, so I began working
with him, meeting weekly on Skype to discuss progress.
}

\begin{frame}

\frametitle{Why was I Interested in Research at CREA?}

\centering
\begin{tabular}{ c | c }
Hobby & CREA Research \\
\hline
Reverse Engineering & Natural Language Processing\\
Game Development & Artificial Intelligence\\
Health & Bioinformatics\\
\end{tabular}

\end{frame}

\section{Understanding Aging}

\begin{frame}

\frametitle{What does CREA want to do?}
\framesubtitle{CREA Wants to Understand Aging and Increase Human Lifespan}

\centering

\includegraphics[width=0.75\linewidth]{images/aging.jpg}

% New discoveries are published quickly and in large volume.
% It is infeasible to construct the knowledge base by hand.
% Working on software to construct the knowledge base automatically.

\end{frame}

\begin{frame}

\frametitle{Why is it Difficult to Understand Aging?}

\centering
\only<1>{\includegraphics[width=.75\linewidth]{images/pubmed.png}}
\only<2>{\includegraphics[width=.5\linewidth]{images/cascadeofbooks.jpg}}

\end{frame}

\begin{frame}

\frametitle{How can CREA Understand Aging?}
\framesubtitle{Build a Knowledge Base that Describes and Reasons about Aging}

\centering
\only<1>{\includegraphics[width=1.0\linewidth]{images/knowledgebase.jpg}}
\only<2>{

\tikzstyle{decision} = [diamond, draw, fill=blue!60,
text width=4.5em, text badly centered, node distance=3cm, inner sep=0pt]
\tikzstyle{block} = [rectangle, draw, fill=blue!60,
text width=5em, text centered, rounded corners, minimum height=4em]
\tikzstyle{line} = [draw, -latex']
\tikzstyle{cloud} = [draw, ellipse,fill=red!60, node distance=3cm,
minimum height=2em]

\begin{tikzpicture}[scale=1, node distance = 2cm, auto]
\node [block] (acquire) {Acquire Biomedical Articles};
\node [block, below of=acquire] (extract) {Extract Knowledge};
\node [cloud, right of=acquire] (machine) {Machine};
\node [cloud, left of=acquire, node distance=4cm] (human) {Human};
\node [block, below of=human] (gui) {Graphical User Interface};
\node [decision, below of=extract] (belongs) {Is about Aging?};
\node [block, left of=belongs, node distance=4cm] (update) {Update Aging Theory};
\node [block, right of=belongs, node distance=4cm] (discard) {Discard};
\path [line] (acquire) -- (extract);
\path [line, dashed] (machine) -- (acquire);
\path [line, dashed] (extract) -| (machine);
\path [line] (extract) -- (belongs);
\path [line] (belongs) -- node{yes}(update);
\path [line] (belongs) -- node{no}(discard);
\path [line, dashed] (human) -- (acquire);
\path [line, dashed] (human) -- (gui);
\path [line] (gui) -- (update);
\end{tikzpicture}

}

% Routinely search for keywords related to aging, dowloading text articles from sources like PubMed.
% Build a spam filter to get rid of non-scientific sentences.
% Extract scientific facts from the sentences and save them in a structured format.
% Provide a graphical interface that allows users to search and explore the knowledge base.

\end{frame}

\section{Knowledge Extraction}

\begin{frame}

\frametitle{Knowledge Extraction}
\framesubtitle{Overview of Progress}

\centering

\begin{block}{Input}
... The \textcolor{orange}{pyridinoline} and \textcolor{orange}{desmosine}
were \textcolor{pink}{examined} as \textcolor{orange}{candidate sensitizer chromophores}
\textcolor{pink}{contained} in \textcolor{orange}{collagen} and \textcolor{orange}{elastin},
respectively. ...
\end{block}

\begin{block}{Output}
\centering
\includegraphics[width=0.5\linewidth]{images/elastinneighborhood.png}
\end{block}

\end{frame}

\begin{frame}

\frametitle{Knowledge Extraction}
\framesubtitle{Method}

\begin{description}[<+->]

\item[Tokenization] Input a text document and read it, one sentence at a time.
\item[Parsing] For each sentence, generate a constituent tree that describes its phrase structure.
\item[Compilation] Extract facts by pattern matching on each constituent tree.

\end{description}

\end{frame}

\begin{frame}

\frametitle{Constituent Tree Tags}
\framesubtitle{Word Level}

\centering

\begin{tabular}{c | c | c}
Tag & Meaning & Example \\
\hline
DT & Determiner & the \\
IN & Preposition & of\\
JJ & Adjective & blue\\
RB & Adverb & quickly\\
CC & Coordinating Conjunction & and \\
NN & Singular Noun & monkey\\
NNS & Plural Noun & monkeys\\
VB & Base Verb & fall\\
VBZ & Singular Present Verb & falls\\
VBD & Past Tense Verb & fell\\
VBN & Past Participle Verb & fallen\\
VBG & Gerund Verb & falling\\
... & ... & ...\\
\end{tabular}

\end{frame}

\begin{frame}

\frametitle{Constituent Tree Tags}
\framesubtitle{Phrase Level}

\centering

\begin{tabular}{c | c | c}
Tag & Meaning & Example \\
\hline
NP & Noun Phrase & the woman \\
VP & Verb Phrase & calls the man \\
PP & Prepositional Phrase & from the store \\
ADVP & Adverb Phrase & quickly and quietly \\
ADJP & Adjective Phrase & blue and red \\
CONJP & Conjunctive Phrase & as well as \\
... & ... & ...\\
\end{tabular}

\end{frame}


\begin{frame}

\frametitle{Constituent Tree Tags}
\framesubtitle{Clause Level}

\centering

\begin{tabular}{c | c | c}
Tag & Meaning & Example \\
\hline
S & Declarative Clause & the dog walks\\
SBAR & Conjunction + Clause & that the dog walks\\
... & ... & ...\\
\end{tabular}

\end{frame}

\begin{frame}

\frametitle{Knowledge Extraction}
\framesubtitle{Example}

\only<1>{
\begin{block}{Input Token}
The man walks the dog.
\end{block}
}

\only<2>{
\begin{block}{Parse Token}
\Tree [.ROOT [.S [.@S [.NP [.DT The ] [.NN man ] ] [.VP [.VBZ walks ] [.NP [.DT the ] [.NN dog ] ] ] ] [.. ] ] ]
\end{block}
}

\only<3>{
\begin{block}{Compile Token}
\Tree [.ROOT [.S [.@S [.NP [.DT The ] [.NN man ] ] [.$\langle$predicate:walk$\rangle$ [.NP [.DT the ] [.NN dog ] ] ] ] [.. ] ] ]
\end{block}
}

\only<4>{
\begin{block}{Compile Token}
\Tree [.ROOT [.$\langle$predicate:walk$\rangle$ [.NP [.DT The ] [.NN man ] ] [.NP [.DT the ] [.NN dog ] ] ] ]
\end{block}
}

\only<5>{
\begin{block}{Compile Token}
\Tree [.ROOT [.$\langle$predicate:walk$\rangle$ $\langle$argument:man$\rangle$ $\langle$argument:dog$\rangle$ ] ]
\end{block}
}

\only<6>{
\begin{block}{Output Fact}
\centering
\includegraphics[width=.5\linewidth]{images/manwalkdog.png}
\end{block}
}

\end{frame}

\begin{frame}

\frametitle{Software Demonstration}
\framesubtitle{A preview of CREA's knowledge base, compiled from PubMed abstracts.}

\href{http://markfarrell.ca/creal}{\includegraphics[width=1.0\linewidth]{images/results.png}}

\end{frame}

\begin{frame}

\frametitle{High Performance}
\framesubtitle{Knowledge can be extracted from many sentences at the same time.}

\includegraphics[width=1.0\linewidth]{images/parallel.jpg}

\end{frame}

\section{What Next?}

\begin{frame}

\frametitle{What Next?}

\centering

\begin{tabular}{c | c}
Task &  Assignees\\
\hline
\textcolor{green}{Research Direction} & \textcolor{green}{Steve Garan} \\
\textcolor{yellow}{Knowledge Extraction} & \textcolor{yellow}{Mark Farrell} \\
\textcolor{yellow}{Text Article Retrieval} & \textcolor{yellow}{Grace Park, Jeremy Wan} \\
\textcolor{yellow}{Knowledge Visualization} & \textcolor{yellow}{Mark Farrell} \\
\textcolor{orange}{Graphical User Interface} & \textcolor{orange}{Mark Farrell} \\
\textcolor{red}{Biomedical Spam Filtering} & \textcolor{red}{---} \\
\textcolor{red}{Automated Reasoning} & \textcolor{red}{---} \\
\end{tabular}


% Filter spam sentences from documents.
% The accuracy of the parser could be optimized:
% Should be trained to identify more nouns from the biomedical domain.
% Define more patterns for extracting facts: The software succeeds around 50\% of the time.


% Support negated clauses and conditional logic.
% Facts can contradict each other:
% Store the probability that is true as the weight of its edge on the
% knowledge base's graph.
% Scale and launch the software service.

% Demonstrated a method for automatically constructing CREA's
% knowledge base on aging.
% Showed how to extract facts from English text in the knowledge
% base's structured format.
% Discussed the need to improve software accuracy by lensing in on
% the biomedical domain.
% Suggested how the software implementation can be scaled for
% production usage.

\end{frame}

\begin{frame}

\frametitle{Get Involved}
\framesubtitle{Contact Me for More Information}

\begin{description}
\item[Email] m4farrel@csclub.uwaterloo.ca
\item[Website] markfarrell.ca
\end{description}

\end{frame}

\end{document}
