\documentclass[mathserif, aspectratio=169]{beamer}
\usetheme{Warsaw}
\useoutertheme{miniframes}
\useinnertheme{rectangles}
\usecolortheme{albatross}
\usepackage{listings}
\usepackage{pgf}
\usepackage{qtree}
\usepackage{tikz}
\usetikzlibrary{shapes, arrows}
\usepackage{gb4e}

\titlegraphic{
\vspace{1.0em}
\includegraphics[height=1.5em]{images/ucberkeley}
\hfill
\includegraphics[height=1.5em]{images/berkeleylab}
}

\title[Understanding the Aging Process with Artificial Intelligence]{Understanding the Aging Process}
\subtitle{An Automated Approach that Uses Artificial Intelligence}

\author{Mark Farrell}
\institute{

Bioinformatics Researcher \and

Center for Research and Education on Aging \\
Lawrence Berkeley National Laboratory \\
University of California, Berkeley

}

\date{September 4th, 2014}
\subject{Bioinformatics}

\note{
Good afternoon everybody. My name's Mark Farrell and I've been doing research for CREA
remotely from Canada. Today I'd like to inform you about a method I've been working on for
automating the construction of biomedical knowledge bases; specifically a knowledge base on aging.
The presentation should be about 20 minutes in length and there will be an opportunity to ask
questions at the end.
}

\AtBeginSection[]
{
\begin{frame}[plain]
\frametitle{Outline}
\tableofcontents[currentsection]
\end{frame}
}

\lstset{
basicstyle=\small\sffamily,
columns=fullflexible,
showstringspaces=false
}

\noautomath

%\setbeameroption{show notes}
\setbeamercolor{note page}{bg=white!90!black, fg=black}

\begin{document}

\begin{frame}[plain]
\titlepage
\end{frame}

\section{About Me}

\begin{frame}

\frametitle{Who I am}
\framesubtitle{Mark Farrell}

\centering

\begin{tabular}{c c}
\includegraphics[width=.2\linewidth]{images/starfall.png}&
\includegraphics[width=.2\linewidth]{images/rink.jpg}\\
\includegraphics[angle=270, origin=c, width=.2\linewidth]{images/alfie.jpg}&
\includegraphics[angle=270, origin=c, width=.2\linewidth]{images/camping.jpg}
\end{tabular}

\end{frame}

\note{
This is my first time being here in person, so I doubt many of you know who I am.
So, I'm going to start off by introducing myself.
}

\begin{frame}

\frametitle{What I'm Studying}
\framesubtitle{Computer Science}

\begin{columns}
\begin{column}{.7\textwidth}
\begin{description}
\item[Year] 2
\item[Program] Bachelor of Computer Science
\item[Faculty] Mathematics
\item[Institution] University of Waterloo
\item[Location] Waterloo, Ontario, Canada
\end{description}
\end{column}
\begin{column}{.3\textwidth}
\includegraphics[width=1.0\linewidth]{images/campus.jpg}
\end{column}
\end{columns}

\end{frame}

\note{
I'm going into my second year of undergraduate studies at the University of Waterloo, majoring in Computer Science.
}

\begin{frame}
\frametitle{Where I'm from}
\framesubtitle{Atlantic Canada}
\centering
\includegraphics[height=.75\textheight]{images/map.png}
\end{frame}

\note{
I grew up on Canada's East Coast, and was back home again this summer.
So, I've had to travel quite a bit over the past few days in order to arrive here at Berkeley.
}

\begin{frame}

\frametitle{How I Became a Researcher at CREA}
\framesubtitle{Prof. Garan Emailed the University of Waterloo Computer Science Club}

\centering

\begin{tabular}{c c}
\includegraphics[width=0.30\linewidth]{images/malto_webcam.png} &
\includegraphics[width=0.30\linewidth]{images/bitshift_webcam.png}\\
\includegraphics[width=0.30\linewidth]{images/csc.png} &
\includegraphics[width=0.30\linewidth]{images/steve_garan.jpg}\\
\end{tabular}

\end{frame}

\note{
That's all very well and good: I haven't explained how I got involved
with research at CREA. Steve Garan (you might know him) emailed the University of Waterloo
Computer Science Club back in January, looking for
current members interested in an AI internship; he used to
be President back in the 80s. It looked interesting, so I began working
with him, meeting weekly on Skype to discuss progress.
}

\section{Understanding the Aging Process}

\note{
Ok, so let's talk about 'Understanding Aging' and where I come in.
}

\begin{frame}

\frametitle{What does CREA want to do?}
\framesubtitle{CREA Wants to Understand Aging and Increase Human Lifespan}

\centering

\includegraphics[height=.5\textheight]{images/aging}

\end{frame}

\note{
As you might already know, CREA wants to understand aging and increase
human lifespan.
}

\begin{frame}

\frametitle{Why is it Difficult to Understand Aging?}
\framesubtitle{High Number of Publications; Growing Rate of Discovery}

\centering
\only<1>{

\includegraphics[width=.95\textheight]{images/pubmed}

\note{
But, Why is it Difficult to Understand Aging?
If you look at this figure here, you can see that not only is there a large
volume of publications made on the subject, but also the
rate at which new publications are being made has been doubling over the past few
decades.
}

}

\only<2>{

\includegraphics[height=.4\textheight]{images/cascadeofbooks}

\note{
So, we need a better way to manage past and future knowledge that will help us understand
aging; this is a problem.
}

}

\end{frame}

\begin{frame}

\frametitle{How can CREA Understand Aging?}
\framesubtitle{Build a Knowledge Base that Describes and Reasons about Aging}

\centering

\tikzstyle{decision} = [diamond, draw, fill=blue!60,
text width=4.5em, text badly centered, node distance=2.5cm, inner sep=0pt]
\tikzstyle{block} = [rectangle, draw, fill=blue!60,
text width=5em, text centered, rounded corners, minimum height=4em]
\tikzstyle{line} = [draw, -latex']
\tikzstyle{cloud} = [draw, ellipse,fill=red!60, node distance=3cm,
minimum height=2em]

\begin{tikzpicture}[scale=1, node distance = 2cm, auto]
\node [block] (acquire) {Acquire Biomedical Articles};
\node [block, below of=acquire] (extract) {Extract Knowledge};
\node [cloud, right of=acquire] (machine) {Machine};
\node [cloud, left of=acquire, node distance=4cm] (human) {Human};
\node [block, below of=human] (gui) {Graphical User Interface};
\node [decision, below of=extract] (belongs) {Is about Aging?};
\node [block, left of=belongs, node distance=4cm] (update) {Update Aging Theory};
\node [block, right of=belongs, node distance=4cm] (discard) {Discard};
\path [line] (acquire) -- (extract);
\path [line, dashed] (machine) -- (acquire);
\path [line, dashed] (extract) -| (machine);
\path [line] (extract) -- (belongs);
\path [line] (belongs) -- node{yes}(update);
\path [line] (belongs) -- node{no}(discard);
\path [line, dashed] (human) -- (acquire);
\path [line, dashed] (human) -- (gui);
\path [line] (gui) -- (update);
\end{tikzpicture}

\note{

And that's where I come in. I've been trying to work on a solution to this problem, which
involves building a knowledge base, software, that can assist humans in order to understand
aging and develop theories about it.

Have a look at this figure: the idea behind the software project I'm working on is to get a machine
to read text and try to understand aging, just like how human's can try understand to aging by
reading texts and journal articles. And also, to present a machine's understanding of aging to humans
and allow them correct and modify the machine's understanding of aging in an intuitive way.
}


\end{frame}

\section{Knowledge Extraction}

\note{
To date, I've made quite a bit of progress in allowing a machine to read and extract knowledge from
text articles, and have also been able to display that knowledge in a manner that helps humans understand aging.
However, I haven't made much progress on allowing the machine to form a theory of aging and reason about aging
itself; i.e. it can't really decide what knowledge is relevant to aging and what it is not.
}

\begin{frame}

\frametitle{What is Knowledge Extraction?}

\centering

\only<1>{
\framesubtitle{Knowledge Extraction is the Act of Reading and Understanding Text}
\includegraphics[height=.75\textheight]{images/knowledgebase.jpg}
}

\only<2>{
\framesubtitle{An Example of Knowledge Extraction}

\begin{block}{Input}
... The \textcolor{orange}{pyridinoline} and \textcolor{orange}{desmosine}
were \textcolor{pink}{examined} as \textcolor{orange}{candidate sensitizer chromophores}
\textcolor{pink}{contained} in \textcolor{orange}{collagen} and \textcolor{orange}{elastin},
respectively. ...
\end{block}

\begin{block}{Output}
\centering
\includegraphics[height=.4\textheight]{images/elastinneighborhood}
\end{block}

}

\note{
As Steve Garan put it, aging can be looked at as the degradation over
time and eventual failure of an organism's systems. So let's start off
talking about knowledge extraction with an excerpt from a journal abstract
that is related to elastin degradation. Have a look at the example input sentence:
I've highlighted the literal nouns found in the sentence and the literal verbs that
relate nouns to one another. Now look at the output: knowledge extraction is the process
of identifying literal nouns and relating them to each other, i.e. making logical assertions
about the actions that they can perform on one another.
}

\end{frame}

\begin{frame}

\frametitle{How is Knowledge Extracted from Text?}
\framesubtitle{An Overview of Knowledge Extraction}

\begin{description}

\item[Tokenization] Input a text document and read it, one sentence at a time.
\item[Parsing] For each sentence, generate a constituent tree that describes its phrase structure.
\item[Compilation] Extract facts by pattern matching on each constituent tree.

\end{description}

\end{frame}

\begin{frame}

\frametitle{How is Knowledge Extracted from Text?}

\only<1>{
\framesubtitle{An Example of How Knowledge is Extracted From Text}
\begin{block}{Input Token - A Sentence}
The man walks the dog.
\end{block}

\begin{block}{Output Fact - An Understanding of the Sentence}
\centering
\includegraphics[height=.25\textheight]{images/manwalkdog}
\end{block}

}

\only<2>{

\framesubtitle{Tokenization: Begin by Reading a Sentence}

The man walks the dog.

}

\only<3>{

\framesubtitle{Parsing: Generate a Constituent Tree for the Sentence}

\tiny
\Tree [.ROOT [.S [.@S [.NP [.DT The ] [.NN man ] ] [.VP [.VBZ walks ] [.NP [.DT the ] [.NN dog ] ] ] ] [.. ] ] ]

}

\only<4>{

\framesubtitle{A Brief Lesson on Parsing Sentences}

\centering
\begin{block}{Assign each Word a Part-of-Speech Tag}
\centering
\tiny
\begin{tabular}{c | c | c}
Tag & Meaning & Example \\
\hline
DT & Determiner & the \\
IN & Preposition & of\\
JJ & Adjective & blue\\
RB & Adverb & quickly\\
CC & Coordinating Conjunction & and \\
NN & Singular Noun & monkey\\
NNS & Plural Noun & monkeys\\
VB & Base Verb & fall\\
VBZ & Singular Present Verb & falls\\
VBD & Past Tense Verb & fell\\
VBN & Past Participle Verb & fallen\\
VBG & Gerund Verb & falling\\
... & ... & ...\\
\end{tabular}

\end{block}

}

\only<5>{

\framesubtitle{A Brief Lesson on Parsing Sentences}

\centering

\begin{block}{Group Words into Phrases}
\centering
\begin{tabular}{c | c | c}
Tag & Meaning & Example \\
\hline
NP & Noun Phrase & the woman \\
VP & Verb Phrase & calls the man \\
PP & Prepositional Phrase & from the store \\
ADVP & Adverb Phrase & quickly and quietly \\
ADJP & Adjective Phrase & blue and red \\
CONJP & Conjunctive Phrase & as well as \\
... & ... & ...\\
\end{tabular}
\end{block}

}

\only<6>{

\framesubtitle{A Brief Lesson on Parsing Sentences}

\centering

\begin{block}{Group Phrases into Clauses - Containing Facts}
\centering
\begin{tabular}{c | c | c}
Tag & Meaning & Example \\
\hline
S & Declarative Clause & the dog walks\\
SBAR & Conjunction + Clause & that the dog walks\\
... & ... & ...\\
\end{tabular}
\end{block}

}

\only<7>{

\framesubtitle{Compilation: Understand the Meaning of the Sentence}

\begin{block}{Find Clauses in the Constituent Tree}
\tiny
\Tree [.ROOT [.\textcolor{red}{S} [.\textcolor{red}{@S} [.NP [.DT The ] [.NN man ] ] [.VP [.VBZ walks ] [.NP [.DT the ] [.NN dog ] ] ] ] [.. ] ] ]
\end{block}

}

\only<8>{

\framesubtitle{Compilation: Understand the Meaning of the Sentence}

\begin{block}{Find Predicates - Actions that Relate Literal Nouns to Form Facts}

\tiny
\Tree [.ROOT [.\textcolor{red}{S} [.\textcolor{red}{@S} [.NP [.DT The ] [.NN man ] ]
[.\textcolor{red}{VP} [.\textcolor{red}{VBZ} \textcolor{red}{walks} ] [.NP [.DT the ] [.NN dog ] ] ] ] [.. ] ] ]
\end{block}

}

\only<9>{

\framesubtitle{Compilation: Understand the Meaning of the Sentence}

\begin{block}{Rewrite the Constituent Tree - Include the Predicate}

\tiny
\Tree [.ROOT [.\textcolor{red}{S} [.\textcolor{red}{@S} [.NP [.DT The ] [.NN man ] ]
[.\textcolor{green}{$\langle$predicate:walk$\rangle$} [.NP [.DT the ] [.NN dog ] ] ] ] [.. ] ] ]
\end{block}

}

\only<10>{

\framesubtitle{Compilation: Understand the Meaning of the Sentence}

\begin{block}{Find Literal Nouns}
\small
\Tree [.ROOT [.\textcolor{red}{S} [.\textcolor{red}{@S} [.\textcolor{red}{NP} [.DT The ] [.\textcolor{red}{NN} \textcolor{red}{man} ] ]
[.\textcolor{green}{$\langle$predicate:walk$\rangle$} [.\textcolor{red}{NP} [.DT the ] [.\textcolor{red}{NN} \textcolor{red}{dog} ] ] ] ] [.. ] ] ]
\end{block}

}

\only<11>{

\framesubtitle{Compilation: Understand the Meaning of the Sentence}

\begin{block}{Rewrite the Constituent Tree - Include Literal Nouns}
\small
\Tree [.ROOT [.\textcolor{red}{S} [.\textcolor{red}{@S} \textcolor{cyan}{$\langle$argument:man$\rangle$}
[.\textcolor{green}{$\langle$predicate:walk$\rangle$} \textcolor{cyan}{$\langle$argument:dog$\rangle$} ] ] [.. ] ] ]
\end{block}

}

\only<12>{

\framesubtitle{Compilation: Understand the Meaning of the Sentence}

\begin{block}{Rewrite the Constituent Tree - Apply Arguments to the Predicate to Form a Fact}

\Tree [.ROOT [.\textcolor{green}{$\langle$predicate:walk$\rangle$} \textcolor{cyan}{$\langle$argument:man$\rangle$}
\textcolor{cyan}{$\langle$argument:dog$\rangle$} ] ]

\end{block}

}

\only<13>{

\framesubtitle{Compilation: Understand the Meaning of the Sentence}

\begin{block}{Output Fact}
\centering
\includegraphics[width=.5\linewidth]{images/manwalkdog}
\end{block}
}

\end{frame}

\begin{frame}

\frametitle{Why is Knowledge Extraction Useful for Understanding the Aging Process?}
\framesubtitle{Let's Examine a Visualization of Knowledge Extracted from PubMed Abstracts to Find Out}

\centering
\href{http://markfarrell.ca/creal}{
\includegraphics[height=.75\textheight]{images/results}
}

\end{frame}

\section{What Next?}

\begin{frame}

\frametitle{What Next?}

\centering

\begin{tabular}{c | c}
Task &  Assignees\\
\hline
\textcolor{green}{Research Direction} & \textcolor{green}{Steve Garan} \\
\textcolor{yellow}{Knowledge Extraction} & \textcolor{yellow}{Mark Farrell} \\
\textcolor{yellow}{Text Article Retrieval} & \textcolor{yellow}{Grace Park, Jeremy Wan} \\
\textcolor{yellow}{Knowledge Visualization} & \textcolor{yellow}{Mark Farrell} \\
\textcolor{orange}{Graphical User Interface} & \textcolor{orange}{Mark Farrell} \\
\textcolor{red}{Biomedical Spam Filtering} & \textcolor{red}{---} \\
\textcolor{red}{Automated Reasoning} & \textcolor{red}{---} \\
\end{tabular}

\end{frame}

\begin{frame}

\frametitle{Get Involved}
\framesubtitle{Contact Me for More Information}

\begin{description}
\item[Email] m4farrel@csclub.uwaterloo.ca
\item[Website] markfarrell.ca
\end{description}

\end{frame}

\begin{frame}[plain]
\centering
Questions?
\end{frame}

\end{document}
